\documentclass{article}
% Document wide customizations
\usepackage[style=phys,
            articletitle=false,
            biblabel=brackets,
            chaptertitle=false,
            pageranges=false,
            url=true,
            hyperref=manual]{biblatex}
\usepackage[en-US,useregional]{datetime2}
\usepackage{etoolbox}
\usepackage{fancyhdr}
\usepackage{import}
\usepackage{kprussing}
\usepackage{listings}
\usepackage{standalone}
\usepackage{titlesec}
\usepackage{titling}
\usepackage{zenburn}
\AtEndPreamble{%
    \usepackage[colorlinks]{hyperref}
    \usepackage[capitalise]{cleveref}
    \BiblatexManualHyperrefOn
}

% Stash if we are using the composite document
\newtoggle{is book}
\makeatletter
\@ifclassloaded{book}{\global\toggletrue{is book}}
                     {\global\togglefalse{is book}}
\makeatother

% Site wide configurations
\renewcommand\familydefault{\sfdefault}
\lstset{upquote=true,
        basicstyle=\small\ttfamily}
\lstdefinestyle{inputs}{%
    style=zenburn,
    emptylines=0,
    breaklines=true,
    postbreak=\mbox{\textcolor{red}{\(\hookrightarrow\)}},
}
% We need to adjust the chapter and bibliography for the full document.
\iftoggle{is book}{%
    \titleformat{\chapter}
                [display]
                {\normalfont\Large\bfseries}
                {\chaptertitlename\ \thechapter}
                {0.5ex}
                {\Large}
                [\large\textnormal{\itshape\theauthor}\\
                       \textnormal{\DTMusedate{current}}]
    \titlespacing{\chapter}{0pt}{-50pt}{20pt}
    \ExecuteBibliographyOptions{refsegment=chapter+}
    \newenvironment{abstract}{%
        \section*{Abstract}
        \small
        \quotation
    }{\endquotation}
}{}

% Configure the header and footer based on inclusion as a chapter or a
% standalone document
\fancyhf{}
\renewcommand\headrulewidth{0pt}
\pagestyle{fancy}
\iftoggle{is book}{%
    \fancyhead[CE]{\rightmark}
    \fancyhead[CO]{\leftmark}
    \fancyfoot[LE,RO]{\thepage}
    \renewcommand{\chaptermark}[1]{%
        \markboth{\chaptername\ \thechapter.\ #1}{}%
    }
    \renewcommand{\sectionmark}[1]{%
        \markright{\thesection.\ #1}%
    }
}{%
    \fancyhead[C]{\rightmark}
    \fancyfoot[R]{\thepage}
}
\fancyfoot[C]{\theauthor{} \textcopyright{} \DTMfetchyear{current}}
\fancypagestyle{plain}{%
    \fancyhead{}
}

\author{Keith F. Prussing, Ph.D.}
\title{The Keith Page}
%\DTMsavedate{current}{\DTMtoday}

% Local macros for document management
\newcommand\blogtitle[1]{%
    \iftoggle{is book}{%
        \chapter{#1}%
    }{%
        \title{#1}%
    }%
}

% Add the bibliography
\addbibresource{references.bib}

\addbibresource{../../references.bib}
\begin{document}
\author{Keith F Prussing, Ph.D.}
\DTMsavenoparsedate{current}{2020}{01}{18}{-1}
\blogtitle{Writing a Data File Parser with Flex and Bison}
\date{\today}
\iftoggle{is book}{}{\maketitle}

\begin{summary}
    So you want to write a parser for a data file for C/C++.  You have a
    plan for a data layout and know it will change slowly.  You have
    read The Art of Unix Programming~\cite{raymond_art_2003} and know a
    plain text format is the best option.  You could hand roll a parser,
    but you are worried about version drift when you need to add a new
    field to the data entry.  Thus, you have decided to use Bison and
    Flex, like a good programmer (a.k.a.  procrastinator), but you do
    not know these tools.  You go to Google only to find that most of
    the tutorials and examples focus on calculators or programming
    languages.  This entry is a walk through of setting up Bison and
    Flex to parse a data file.
\end{summary}

\section{Problem Statement}

The basic problem is we want to pull data from a file.  In the program,
we want to represent the data as a hash table of objects that could be a
string for keying other details (like a rendering model) or a sequence
of floats.  The main program is written in C++ and we do not want to add
the overhead of adding Haskell as a language dependency, even though it
would give us a clean way to deal with missing data.  Instead, we define
the internal storage using a \lstinline[language=C++]{std::map} and a
new \lstinline[language=C]{struct}\footnote{The full source for shit
post is available on \href{https://github.com/kprussing/kprussing.github.io/tree/latex-migration/posts/2020-01-18-writing-a-parser}{github}.}

\begin{lstlisting}[language=C]
#include <map>
#include <string>
#include <vector>

typedef enum Type {
    Nothing,
    String,
    Floats
} Type;

typedef struct Datum {
    Type                type;
    std::string         string;
    std::vector<float>  data;
} Datum;

std::map<std::string, Datum> database;
\end{lstlisting}

Looking at this representation, we see a few important points.  First,
the data are keyed by a unique string.  Second, we need a field to
toggle string versus floating point data.  And third, the data can be of
arbitrary length.  To be nice, we would like to allow the user
optionally specify the length of the floating point data for validating
data, and we would like the default behavior to be reading a single
floating point value if no toggle string is given.  This leads us to a
regular expression representation of the form:

\begin{lstlisting}[language=Python]
'(?P<uid>\w+):(?:string|float(?:[:](?P<len>\d+))?):(?P<vals>.*)'
\end{lstlisting}

For a few other niceties, we would like to:

\begin{itemize}
    \item   Allow comments to the end of line after the \lstinline{#}
        for anything other than the string values.
    \item   Ignore case in the unique id (\lstinline{uid}) to avoid easy
        editor bobbles.
    \item   Ignore white space around the field separator
        (\lstinline{:}).
    \item   And allow the trailing floating point values to span lines.
\end{itemize}

\section{Lexical Analysis with Flex}

Flex~\cite{westes_flex_2001} is the lexical analyzer
replacing the venerable \lstinline{lex}.  Its task is to evaluate
sequences of characters and perform a task (most commonly return a
token).  Flex uses regular expressions to identify the tokens (yay!),
but it know extended regular expressions not Perl style regular
expressions (boo…).  That means we need to add a two more constraints
because the grouping task harder for strings:

\begin{itemize}
    \item   Unique ids cannot contain spaces.  They must be valid C
        style identifiers (plus \lstinline{-}).
    \item   The string value cannot contain spaces.
\end{itemize}

Our first task is to come to grips with Flex.  We can follow along with
the tutorial by verBurg~\cite{verburg_flex_2018} to build the
functionality.  Adapting the first Flex example we have

\begin{code block}
    \lstinputlisting[language=C, style=inputs]{flex-only/flex-only.l}
\end{code block}

We have two major changes compared to the example in the tutorial.
First, we abstracted the definition of a floating point value to shorten
the regular expression in the token definitions.  This allows us to
accept the common ways to write a floating point value.  Second, we
introduced the comment character \lstinline{#} to add comments to the
end of line.  To build, we could do things by hand, but a better
solution is to use CMake~\cite{kitware_cmake_2019}.  The
\lstinline{CMakeLists.txt} for our simple project starts out with:

\begin{code block}
    \lstinputlisting[style=inputs]{flex-only/CMakeLists.txt}
\end{code block}

The important point to note is the use of \lstinline{CMAKE_CURRENT_BINARY_DIR}
to make sure the source gets placed in the correct location (i.e. no
build products in the source tree).  At this point, you can build and
test the example (how to do that is a quick Google search away);
however, this example is not exactly thrilling.  Now that we have the
token analyzer, we can turn to the actual parsing.

\section{Parsing with Bison}

Our first task is to update the Flex input to return values.  For each
data type of interest, we use the standard library functions to convert
the text to a usable type.  For strings, we copy the contents of the
pointer because Flex can potentially overwrite the data before we can
use it.  Note, \emph{we} are responsible for \lstinline{free}ing the
memory once we're done with it in the calling code.  We assign these to
values in the \lstinline{yylval} structure which is declared in
\lstinline{example.parser.hh} which will be emitted by Bison.

\begin{code block}
    \lstinputlisting[style=inputs]{bison-1/lexer.l}
\end{code block}

Next, we update the \lstinline{CMakeLists.txt} to prepare for the call
to Bison.  One annoying part of using CMake for this task is we have to
explicitly tell it that the Flex file depends on the result of the call
to Bison (even though it knows how to scan inputs.  I bet SCons could do
it right…).  But, it's better than managing a \lstinline{Makefile}
directly.  Note, we are using relative paths for the flex input only for
example purposes.  \emph{Do not} take this as the ``right'' way to setup
a CMake project.

\begin{code block}
    \lstinputlisting[style=inputs]{bison-1/CMakeLists.txt}
\end{code block}

Now we are ready to write the Bison file.  Bison is a Yacc-compatible
parser generator~\cite{gnu_bison_2019}.  It takes a sequence of tokens
and assembles the parsing table.  We begin with another simple example
that just prints what the parser sees.

\begin{code block}
    \lstinputlisting[style=inputs]{bison-1/parser.y}
\end{code block}

The important things we see here are the declaration of the symbols
we'll need when parsing and in the copied code blocks.  We have kept the
main function in the copied code block because the header section (the
code between the \lstinline[language=C]||)
is \emph{not} copied to the header.  Thus, we cannot just include the
header in a separate main file.  We have the usual definition of the
Bison union and the terminal tokens.

The major change from verBurg's tutorial is the formatting.  Reading
over the Bison manual~\cite{gnu_bison_2019}, they note the white space
does not matter, but they recommend this format with the vertical bar
`|' in the first column.  Upon seeing this, it dawned on me that the
rules just work out to functional programming like Haskell.  Each of the
targets before the colon is a non-terminal symbol that is comprised of
the sequence to the right of the colon.  They can be recursively defined
(left recursive tends to be better) or they can be comprised of terminal
symbols (\lstinline{INT}, \lstinline{FLOAT}, or \lstinline{STRING} in
this case).

If we run the code with the example data file, we see it just prints the
data from the file.  The parsing ignores the comments and blank lines
like we want.  We do see the colon, our designated field separator, gets
printed to the standard output.  Our next task is to build up the rules
that will collect the values into collections and associate them with a
property.

\begin{code block}
    \lstinputlisting[style=inputs]{example.dat}
\end{code block}

\section{Separating the Fields from the Values}

Now that we have a feel for what Bison is doing, we can expand the
example to separate the fields from the values.  The first step is to
add the following lines to the Flex input to help with error reporting
and ensuring the colon is returned to the parser.

\begin{code block}
    \lstinputlisting[style=inputs]{bison-2/lexer.l.patch}
\end{code block}

The next step is to update the parser to collect the property names and
the values as shown below.  The key changes are declaring storage in the
initial header (the strings and vector) and storing the data
appropriately in either the key or value strings or in the vector
storage.

\begin{code block}
    \lstinputlisting[style=inputs]{bison-2/parser.y}
\end{code block}

An important thing to note is the arguments to the output (the dollar
sign followed by a number) include counting the actions inside the
braces.  Thus, the value to assign to \lstinline{val} is \lstinline{4}
not \lstinline{3}.  This gets the assignment correct and properly frees
the memory.

\section{Assembling the Output}

Now we put it all together.  We have the keys and the values that we can
assemble into the main \lstinline{Database}.  For the Flex input, we
symbolically define the separator and symbolic references for the
different datum types.

\begin{code block}
    \lstinputlisting[style=inputs]{lexer.l}
\end{code block}

For the Bison input, we have more to do.  We remove the main routine and
place it in a separate input file to demonstrate loading the data as a
separate library.  We also define additional variables for tracking the
desired output shape of the vector data and new tokes for triggering the
types.  Next, we begin to separate the semantics to grab the key and
then separate the logic for the different types.  A string type is
simply the string value while the scalar and vector processes through
the possibly defined shapes before we process the values.  After reading
all the values, we reshape the data and place it in the output.

\begin{code block}
    \lstinputlisting[style=inputs]{parser.y}
\end{code block}

Note, we defined the database as a global variable.  This is sufficient
for a single parser with a centrally maintained database, but it will
not work if we need multiple parsers or to manage multiple databases.

The final step is updating the CMake input which should be a straight
forward task.

\begin{code block}
    \lstinputlisting[style=inputs]{CMakeLists.txt}
\end{code block}

\section{Conclusions}

In this post, we outlined how to get Bison and Flex working in yet
another tutorial.  The presented implementation works for a centrally
managed database, but it will probably not work if multiple databases
are necessary.  The ``correct'' solution would be to use a reentrant
parser, but I wasn't able to get that working (I couldn't get Flex and
Bison to both see the class definition and match the function
signatures).  But, after reflecting on what I actually need for loading
data in my use case, I decided that going full YAML (or the equivalent)
will serve me better.  So with that, I'll leave getting the reentrant
parsing for another day.

\bibliographystyle{apsrev4-2}
\bibliography{references.bib}
\end{document}
